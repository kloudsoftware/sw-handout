\chapter{Sicherheit}
\section{Architektonische Schwächen}

  \begin{itemize}
    \item Die meisten Clients ignorieren CRL's wenn sie kein Internet haben
    \item Root Zertifikate können nicht widerrufen werden
  \end{itemize}

    

  \begin{itemize}
    \item Delegation Problem
        \begin{itemize}
            \item untergeordnete CA's können nicht vom Ausstellen von Zertifkaten außerhalb eines Namespaces abgehalten werden
            \item dadurch existieren sehr viele CA's
            \item Klassifizierung dieser nicht möglich
        \end{itemize}
  \end{itemize}

    

  \begin{itemize}
    \item Federation Problem
        \begin{itemize}
            \item untergeordnete CA's und cross signing machen Validierung komplex
            \item braucht sehr viel CPU Zeit
        \end{itemize}
  \end{itemize}

    
\section{Probleme mit CA's}

  \begin{itemize}
    \item CA's bieten so gut wie keine Gewährleistungen
    \item Expiration Date wird benutzt um den User für eine Erneuerung zur Kasse zu bitten, anstatt zur Regulierung der Stärke des Schlüssels
    \item CA's missbrauchen Sicherheitsmerkmale der Zertifikate aus ökonomischen Gründen
  \end{itemize}



  \begin{itemize}
    \item Details von CSR's sind von CA zu CA unterschiedlich, meist keine einfache Erklärung
    \item CA's sind Regierungen untergeordnet, dies widerspricht den Interessen der User
  \end{itemize}


\section{Implementations Schwächen}

  \begin{itemize}
    \item Widerrufsprüfung oft ignoriert
        \begin{itemize}
            \item werden als Hürde gesehen
            \item Infrastruktur nicht solide genug
        \end{itemize}
  \end{itemize}



  \begin{itemize}
    \item Key usage Erweiterung oft ignoriert
    \item Bestimmte Erweiterungen als kritisch zu markieren crasht clients
    \item Code Injection Attacks
  \end{itemize}


\section{Kryptographische Schwächen}

  \begin{itemize}
      \item Sicherheit von Zertifkaten baut auf Sicherheit der Hash Funktion auf
      \item Kollisionen führen zu Angriffsvektoren
      \item Angreifer können beliebige Zertifikate im Namen einer CA austellen
  \end{itemize}

